\documentclass{article}
\title{\textbf{The role of Artificial Intelligence in future technology}}
\author{Johannes Weidmann  \\
	\small{Scientific essay for the application at the Technical University of Munich}
}

\date{\today}
% Hint: \title{what ever}, \author{who care} and \date{when ever} could stand 
% before or after the \begin{document} command 
	% BUT the \maketitle command MUST come AFTER the \begin{document} command! 
\begin{document}

\maketitle

Artificial Intelligence (AI) refers to the development and implementation of computer systems capable of performing tasks that typically require human intelligence. It encompasses a wide range of techniques and algorithms that enable machines to mimic cognitive processes such as learning, problem-solving, perception, and decision-making. It has emerged as a transformative force in the field of technology, revolutionizing various sectors and paving the way for innovative solutions. With its ability to mimic human intelligence and learn from data, AI has the potential to reshape the future of technology. By examining the current state of AI and analyzing its applications in different domains, we can understand the vast potential and challenges associated with this technology. \par \ \par

The pioneering work of British mathematician and computer scientist Alan Turing can be credited with giving rise to artificial intelligence (AI). The "Turing Test" was a groundbreaking idea that Turing put forth in 1950 \cite{1}.
The Turing Test was created to ascertain whether a machine could behave intelligently and be mistaken for a person. According to Turing, a computer would be deemed intelligent if it could have a discussion with a human judge and succeed in making them believe it was a person. \par
In the 1980s, machine learning techniques, in particular neural networks, started to gain popularity. The backpropagation method was first described by Rumelhart et al. in 1986 \cite{2}. This algorithm enabled efficient training of neural networks, leading to significant improvements in pattern recognition and speech processing tasks.
Deep learning, a subset of machine learning, gained prominence in the 2010s due to its ability to extract hierarchical representations from data. Object detection algorithms that are fast and accurate and offer real-time object recognition establish many possibilities e.g. self-driving cars without complex sensors or real-time information of assistive devices for human users \cite{3}. 
Large-scale language models have proven essential in the recent developments in natural language processing. The GPT (Generative Pretrained Transformer) model was first introduced by OpenAI in 2018, followed by GPT-1, GPT-2, and GPT-3. The potential for chatbots, text completion, and language translation jobs are expanded by these models' outstanding abilities in language understanding and generation \cite{4}. \par \ \par

The healthcare sector is one of AI's most potential uses. Medical diagnostics, drug discovery, and tailored medicine could all be revolutionized by AI. AI systems can find trends and forecast disease outcomes with greater accuracy than conventional approaches by analyzing massive datasets and medical records. Today, algorithms already surpass radiologists in identifying cancerous tumors and advising researchers on how to create cohorts for expensive clinical trials. Deep learning could be used in the diagnosis to identify skin diseases or cancer like malignant tumors in radiology pictures \cite{5}. Therefore, in the future, more remote diagnoses will be possible, so that more doctors, instead of diagnosing people, will be able to do more clinical research \cite{7}. Additionally AI could also help reduce waiting times in doctors' offices by replacing patient management with an AI chatbot \cite{8}. This could potentially go as far as AI taking the patient's medical history, which would result in less waiting time and improve overall health. AI will also help to end or even prevent pandemics, such as the Covid 19 pandemic, more quickly. Neural networks can help predict, analyze and visualize protein structures of viruses \cite{6}. \par 

AI has the potential to completely change the art and music industries. The use of AI algorithms and machine learning methods can improve the creative process, opening up new avenues for artistic expression. AI-based systems have demonstrated their ability to create and compose music. It is possible to look at completely new facets of music. In 2020, for example, the official anthem of the Olympic Games in Tokyo was generated by artificial intelligence \cite{11}. It will therefore also be possible for us to listen to music in the future based on our emotional state. AI assistants could thus support people in stressful situations by generating suitable music. \par

AI is expected to have a big impact on transportation in the future. Autonomous vehicles, powered by AI algorithms, could change the way we travel. They can analyze sensor data, make quick decisions, and navigate traffic. These vehicles could improve safety, reduce traffic, and save energy. An AI controlled car has the advantage of not allowing distraction and fatigue, even after hours of driving, unlike humans. It can also make traffic flow more smoothly, as there is no delay in starting due to human reaction time. It also has the effect of enabling disabled or elderly people to remain independent and mobile \cite{9}.
AI can also optimize processes in the supply chain of companies, which will eventually run fully automated. In the future, it will be possible to communicate with IT systems via voice. Intelligent WMS systems will also be able to recognize how objects should be optimally stored in order to increase sales. The products can then be shipped autonomously by self-driving trucks \cite{10}. \par \ \par

While AI presents tremendous opportunities, it also raises ethical and social concerns. One major concern is the potential displacement of jobs due to automation. As AI systems become more capable, there is a risk that certain job roles may become obsolete. While a company could use a AI driven robot for logistics they have no reason to hire human employees \cite{10}. \par 
Another risk is the use of deep fakes in social media. Today, it is easier than ever to spread rumors or fake identities. Lip sinking, the mapping of a person's audio and video, allows statements that were never said to be sold as true \cite{12}. Thus, it will become increasingly difficult to distinguish between truth and lies. \par 
Another problem with artificial intelligence is that the more decisions are made by it, the more ethical issues arise. The trolley problem is an ethical thought experiment in which a person is faced with the choice of either causing the death of one person in order to save several lives, or remaining inactive and thus accepting the death of several people. By enabling autonomous driving through training by data, such problems must be well thought out \cite{13}. \par \ \par

The potential for artificial intelligence to influence future technology in many different fields is enormous. Its skills in the healthcare, transportation, culture and other industries hold the potential of increased productivity, increased safety, and improved decision-making. But just like any other game-changing technology, AI also poses moral and social issues that need to be properly considered. We can leverage the value of AI while reducing its risks by encouraging collaboration between researchers, politicians, and industry leaders.

\newpage

\begin{thebibliography}{9}
	\bibitem[1]{1} \emph{A. M. Turing, “Computing machinery and intelligence”, 1950}, 
	\bibitem[2]{2} \emph{David E. Rumelhart et al., "Learning representations by back-propagating errors", 1986},
	\bibitem[3]{3} \emph{Joseph Redmon et al., “You Only Look Once:
		Unified, Real-Time Object Detection”, 2016},
	\bibitem[4]{4} \emph{A. Radford et al., "Improving Language Understanding
		by Generative Pre-Training", 2018},
	\bibitem[5]{5} \emph{Thomas Davenport and Ravi Kalakota, "The potential for artificial intelligence in healthcare", 2019},
	\bibitem[6]{6} \emph{A.W. Senior et al., "Improved protein structure prediction using potentials from deep learning", 2020},
	\bibitem[7]{7} \emph{M. Ghassemi et al., "A Review of Challenges and Opportunities in Machine Learning for Health", 2018},
	\bibitem[8]{8} \emph{S. Sunarti et al., "Artificial intelligence in healthcare: opportunities and risk for future", 2020},
	\bibitem[9]{9} \emph{Bhavesh Ajaykumar Shukla et al., "Research Paper on AI in Driving", 2022},
	\bibitem[10]{10} \emph{DHL Customer Solutions and Innovation, "ARTIFICIAL INTELLIGENCE
		IN LOGISTICS: A collaborative report by DHL and IBM on implications
		and use cases for the logistics industry", 2018},
	\bibitem[11]{11} \emph{Miguel Civit et al., "A systematic review of artificial intelligence-based music generation: Scope, applications, and future trends", 2022},
	\bibitem[12]{12} \emph{T. Brooks et al. , "Increasing Threats of Deepfake Identities"}
	\bibitem[13]{13} \emph{Maximilian Geisslinger, "Autonomous Driving Ethics: from Trolley Problem to Ethics of Risk", 2021}
\end{thebibliography}

\end{document}